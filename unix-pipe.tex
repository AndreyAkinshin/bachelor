Консольные приложения --- одна из важнейших вех в истории развития
компьютеров. До появления графических интерфейсов консоль была
единственным средством общения с ЭВМ и даже сейчас, в эпоху победившего GUI,
консоль никуда не исчезла. Конечно, консольные приложения, как
правило,  не обладают таким красивым и приятным для глаз интерфейсом.
Не  всегда сразу понятно как ими пользоваться, особенно для неискушенных
пользователей, --- приложения с GUI более очевидны для восприятия. Однако, у
консольных приложений есть некоторые неоспоримые
преимущества, которые отсутствуют у графических интерфейсов:

\begin{enumerate}
\item{Скорость работы с CLI гораздо выше}
\item{UNIX PIPE}
\end{enumerate}

Рассмотрим пункт №2 подробнее.

Конвейер в терминологии UNIX — некоторое множество процессов, для
которых выполнено следующее пере направление ввода-вывода: то, что
выводит на поток стандартного вывода предыдущий процесс, попадает в
поток стандартного ввода следующего процесса.

Механизм конвейеров добавляет гибкость процессу взаимодействия программ.
Все программы оперируют текстовыми данными, т.е. получают на вход текст и
выводят его же, причем, обработка данных внутри программы идет <<в
потоке>>, т.е. без лишнего копирования и хранения, что существенно
повышает производительность \cite{habr:gulp}. Данные как-бы <<протекают>>  через
цепочку программ, разделенных символом конвейера. Таким образом, имея
некоторый набор простых утилит для обработки текста в потоке,
пользователь может <<собрать>> себе новую программу под свои задачи,
просто объединив конвейером несколько программ с параметрами: 

\begin{lstlisting}
    tail -f /var/log/Xorg.0.log | grep EE
\end{lstlisting}

Таким образом, CLI-интерфейс является одним из наиболее гибких средств для
создания прикладных, и, в особенности, служебных приложений, а также предоставляет
удобнейшее средство для применения модульного подхода к построению
архитектуры, состоящей из множества отдельных приложений.

Такой интерфейс приложений востребован среди пользователей *nix систем. Поэтому
создание консольного плеера для Soundcloud является актуальной
задачей. 
