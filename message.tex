\subsection{Структура команд сервера}

Клиент формирует запросы серверу в виде команд. Структура команды
запроса к серверу содержит в себе название команды и список параметров в
формате JSON: 

\begin{jsoncode}
  { 
    "name": "play", 
    "params": {
      "tag": "rock"
    } 
  }
\end{jsoncode}

При создании сервера определяется набор команд, реализующий функции
проигрывания. Серверная команда включает в себя поля, идентичные полям команды запроса,
описания параметров и функцию к исполнению:

\begin{jscode}
  playIndex: {
    name: "playIndex",
    params: { index: "index of playlist element"},

    exec: function (params) {

      var index = params.index * 1 || 0;
      
      model._currentTrackIndex = index;            
      daemon.play();

      return 0;
      
    }
  },

\end{jscode}

Функция \textit{exec} обязательно должна возвращать результат --- это
может быть либо результат запроса, либо статус выполнения
команды. В случае, если запрашиваемой команды не существует, клиенту
возвращается код ошибки и список доступных команд.