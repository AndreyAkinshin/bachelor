\addcontentsline{toc}{section}{Приложение \Asbuk{section} Задание на
  бакалаврскую работу} 

\append{Руководство пользователя}

Приложение предназначено для исполнение в среде текстовой консоли, которая обычно
предоставляется эмулятором терминала. Для запуска необходимо знать
сетевой адрес, на котором запущен сервер. Если это тот же компьютер,
то достаточно выполнить команду:

\begin{bashcode}
  ncs-tui localhost
\end{bashcode}

В противном случае необходимо заменить \textit{localhost} на адрес
сервера воспроизведения.

Окно программы содержит в себе элементы отображения статуса
проигрывателя и строку ввода команд. В окне программы можно увидеть
(снизу вверх)

\begin{itemize}

\item{ текущую временную отметку}
\item{ длительность текущей композиции, }
\item{ её название,}
\item{ автора, }
\item{ область заголовка текущего раздела,}
\item{ область текущего раздела,}
\item{ индикатор прогресса композиции}
\item{ строку ввода команд }

\end{itemize}

\addimghere{nsc-playlist}{0.6}{Элементы интерфейса окна программы}

Интерфейс имеет четыре логических раздела:
\begin{enumerate}
\item{ помощь }
\item{ текущий плейлист }
\item{ информация о композиции }
\item{ комментарии }
\end{enumerate}

Переход по разделам осуществляется по нажатию соответствующим им
номерным клавишам 1,2,3,4 на клавиатуре. Для удобства
навигации при запуске программы область заголовка раздела отображает
помощь по навигации:  

\addimghere{nsc-launch.eps}{0.6}{Запуск программы}

В разделе помощи всегда можно найти краткую справку по управлению
проигрывателем: 

\addimghere{nsc-help}{0.6}{Раздел помощи}

\newpage

Во вкладке <<плейлист>> отображается список воспроизведения текущих
композиций, расположенных в порядке проигрывания. Здесь можно сменить
проигрываемую композицию, выбрав нужную из списка и нажав клавишу
Enter:

\addimghere{nsc-playlist}{0.6}{Раздел списка воспроизведения}

Вкладка информации о композиции содержит в себе краткий обзор свойств
проигрываемой композиции:

\addimghere{nsc-track-info}{0.6}{Раздел информации о композиции}

Раздел комментариев представляет собой список отзывов пользователей,
появляющихся по мере проигрывания композиции: 

\addimghere{nsc-comments}{0.6}{Раздел комментариев}

Для начала воспроизведения радио необходимо нажать сочетание клавиш
Alt-x для отображения строки ввода команд. В ней необходимо выполнить
команду \textit{play} с желаемым жанром для воспроизведения.

\addimghere{nsc-cmd-prompt}{0.6}{Строка ввода команд}

После этого проигрыватель загрузит список композиций плейлист и начнет
воспроизведение. Также для выполнения доступны команды \textit{pause}
и \textit{resume}, останавливающие и возобновляющие воспроизведение
текущей композиции.

