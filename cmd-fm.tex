\subsection{Web-приложение Cmd.fm}

Главным и практически единственным мощным аналогом является сайт
\textit{сmd.fm} \cite{cmd-fm:main}. Этот сайт реализует радио на основе доступных на
\textit{Soundcloud} композиций. Интерфейс реализован в виде
web-приложения и представляет собой имитацию текстового терминала.

\addimghere{cmdfm-comments}{0.5}{Страница комментариев cmd.fm}

Команда play запускает радио, выбрав в плейлист
композиции, соответствующие указанному жанру. Также доступна
интерактивная  справка по командам и другие элементы, характерные для
каноничных приложений с текстовым интерфейсом. 

Многим пользователям этого сайта очень понравилась \cite{stubler:cmdfm}
реализация основной особенности \textit{Soundcloud} --- комментариев к
композиции: по мере проигрывания в консоль плеера выводятся
комментарии, оставленные пользователями в этот момент. 

Долгое время пользователи \textit{cmd.fm} ждали появления аналогичного
приложения в формате текстового приложения для эмулятора
терминала. В данный момент проект находится на стадии
β-тестирования и доступно только для платформы MacOS X. Код
приложения не доступен в открытом доступе, поэтому изменить
архитектуру приложения на клиент-серверную не представляется возможным.