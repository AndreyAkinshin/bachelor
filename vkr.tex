\section{Описание предметной области}
\subsection{Постановка задачи}

На сайте Soundcloud каждый исполнитель
может выложить свои композиции для всеобщего прослушивания и
скачивания. Он отличается от всех музыкальных сайтов тем, что
позволяет привязывать комментарий не только к целой композиции, а к
отдельным ее частям. На каждой web-странице композиции есть виджет
музыкального плеера для её проигрывания. 

На виджете в  местах, соответствующих временным отметкам,
расположены иконки пользователей, оставивиших в этом месте
комментарий. В момент, когда проигрываемая композиция проходит отметку
с иконкой, над ней показывается комментарий. Таким образом,
прослушивая трек можно узнать о мнениях пользователей не только насчет
всей композиции, но и насчет отдельных её моментов. Это очень
актуально не только для музыкантов и продюссеров, но также интересно и
для обычных слушателей. 

Сервис предоставляет удобное API и SDK для доступа к данным сайта, что
позволяет создавать приложения на любой платформе. 

{\itshape описание кучи клиентов разных тут должно быть}

Однако, в классе консольных приложений представлен только один CLI-клиент, реализующий
лишь базовые функции, без возможности комментирования. Данный
вид приложений востребован среди пользователей *nix систем. Поэтому
создание консольного плеера для Soundcloud является актуальной
задачей. 

Консольные приложения - одна из важнейших вех в истории развития
компьютеров. До появления графических интерфесов консоль была
едиственным средством общения с ЭВМ и даже сейчас, в эпоху победившего GUI,
консоль никуда не исчезла. Конечно, консольные приложения, как
правило,  не обладают таким красивым и прятным для глаз интерфесом.
Не  всегда сразу понятно как ими пользоваться, особенно для неискушенных
пользователей, --- приложения с GUI более очевидны для восприятия. Однако, у
консольных приложений есть некоторые неоспоримые
преимущества, которые отстутствуют у графических интерфейсов:

\begin{enumerate}
\item{Скорость работы с CLI гораздо выше}
\item{UNIX PIPE}
\end{enumerate}

Рассмотрим пункт №2 подробнее.

Конвейер в терминологии UNIX — некоторое множество процессов, для
которых выполнено следующее перенаправление ввода-вывода: то, что
выводит на поток стандартного вывода предыдущий процесс, попадает в
поток стандартного ввода следующего процесса.

Механизм конвееров добовляет гибкость процессу взаимодействия программ.
Все программы оперируют текстовыми данными, т.е. получают на вход текст и
выводят его же, причем, обработка данных внутри программы идет <<в
потоке>>, т.е. без лишнего копирования и хранения, что существенно
повышает производительность \cite{habr:gulp}. Данные как-бы <<протекают>>  через
цепочку программ, разделенных символом конвеера. Таким образом, имея
некоторый набор простых утитилит для обработки текста в потоке,
пользователь может <<собрать>> себе новую программу под свои задачи,
просто объеденив конвеером несколько программ с параметрами: 

\begin{lstlisting}
    tail -f /var/log/Xorg.0.log | grep EE
\end{lstlisting}

Таким образом, CLI-интерфейс является одним из наиболее гибких средств для
создания прикладных, и, в особенности, служебных приложений, а также предоставляет
удобнейшее средство для применения модульного подхода к построению
архитектуры, состоящей из множества отдельных приложений.

Сервис предоставляет удобное API и SDK для доступа к данным сайта, что
позволяет создавать приложения на любой платформе. Однако, в классе
консольных приложений представлен только один CLI-клиент, реализующий
лишь базовые функции, без возможности комментирования. Однако,  данный
вид приложений востребован среди пользователей *nix систем. Поэтому
создание консольного плеера для Soundcloud является актуальной
задачей. 

