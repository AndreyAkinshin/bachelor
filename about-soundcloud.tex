\subsection{Сервис Soundcloud и его особенности}

На сайте {\itshape Soundcloud} каждый исполнитель
может выложить свои композиции для всеобщего прослушивания и
скачивания. Он отличается от всех музыкальных сайтов тем, что
позволяет привязывать комментарий не только к целой композиции, но и к
отдельным ее частям.

На каждой web-странице композиции есть виджет музыкального  для её
воспроизведения. 

\addimghere{player-widget.png}{0.7}{Виджет плеера на Soundcloud}

На виджете в местах, соответствующих временным отметкам
расположены иконки пользователей, оставивших в этом месте
свои комментарии. В момент когда проигрываемая композиция проходит отметку
с иконкой пользователя над ней показывается сответствующий этой
отметке комментарий. Таким образом,
прослушивая трек можно узнать о мнениях пользователей не только насчет
всей композиции, но и об отдельных её частях. 
Это не только очень актуально для музыкантов и продюсеров, но также
интересно и для обычных слушателей.

Сервис предоставляет удобное API и SDK для доступа к данным сайта, что
позволяет создавать приложения на любой платформе. И существует
множество  мобильных, настольных и web-приложений, которые
пользуются данными сервиса и предоставляют доступ к управлению его содержимым.  

Рассмотрим приложения, которые больше всего подходят под выдвинутые во введении
требования.