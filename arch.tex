\section{Архитектура приложения}

Разработанный плеер имеет клиент-серверную архитектуру, реализующую
паттерн MVC: серверная служба будет выступать в роли
контроллера, модель -- компонент для связи с API, а клиенты выступают в роли
представлений. 

Model-view-controller (MVC, «модель-представление-поведение») --- схема
использования нескольких шаблонов проектирования, с помощью которых
модель данных приложения, пользовательский интерфейс и взаимодействие
с пользователем разделены на три отдельных компонента таким образом,
чтобы модификация одного из компонентов оказывала минимальное
воздействие на остальные.\cite{wiki:mvc}

Такая архитектура выбрана из соображений расширения и
масштабируемости, так как позволяет создавать новые клиенты на разных
платформах абстрагируясь от реализации доступа к данным
сервиса. Именно это является ключевым отличием от немногочисленных
аналогов, имеющих монолитную архитектуру и привязанных к одной
платформе.

\addimghere{nsc-general}{0.3}{Общая архитектура системы}

На уровне архитектуры не предусмотрено механизмов разграничения прав доступа. Это
сделано для упрощения модели: предполагается, что класс приложений,
построенных на этой архитектуре, не нуждается в правах доступа к
ресурсам. В случае необходимости их можно реализовать на уровне
модели.

\subsection{Сетевое взаимодействие компонентов системы}

Сервер и клиенты общаются между собой по сети. Когда в сети появляется
новый клиент, он подписывается на интересующие его события
проигрывателя, такие как:

\begin{itemize}
\item{ началось воспроизведение новой композиции }
\item{ сменился список воспроизведения }
\item{ сменилась временная позиция }
\end{itemize}

Таким образом, взаимодействие между клиентами и сервером происходит по
принципу \textit{publish/subscribe}.

% \addimghere{publish-subscribe}{0.6}{Взамодействие по протоколу публикация/подписка}

Когда клиент только появляется в сети или запрашивает данные,
взаимодействие между клиентом и сервером осуществляется по принципу
\textit{socket} ---  двусторонняя связь между новым клиентом и
сервером.

По команде \textit{getCommandList} или при запросе несуществующей
команды сервер возвращает код ошибки и список доступных команд. Такое
архитектурное решение было выбрано с целью обеспечить возможность
легкой расширяемости системы команд сервера. С помощью списка команд
клиент узнаёт о возможностях сервера и может решить, какие функции
будут доступны в интерфейсе, а какие нужно убрать. Например, если
сервер не отправляет информации о текущей временной позиции, то
зависящие от неё элементы можно исключить или выдать сообщение, что
запуск клиента не возможен. В то же время простым клиентам, которые
могут только переключить композицию на следующий эта функция вообще не
известна, они работают и без неё --- им нужно лишь информация о
доступности команды \textit{next} на сервере.

\subsection{Структура команд сервера}

Клиент формирует запросы серверу в виде команд. Структура команды
запроса к серверу содержит в себе название команды и список параметров в
формате JSON: 

\begin{jsoncode}
  { 
    "name": "play", 
    "params": {
      "tag": "rock"
    } 
  }
\end{jsoncode}

При создании сервера определяется набор команд, реализующий функции
проигрывания. Серверная команда включает в себя поля, идентичные полям команды запроса,
описания параметров и функцию к исполнению:

\begin{jscode}
  playIndex: {
    name: "playIndex",
    params: { index: "index of playlist element"},

    exec: function (params) {

      var index = params.index * 1 || 0;
      
      model._currentTrackIndex = index;            
      daemon.play();

      return 0;
      
    }
  },

\end{jscode}

Функция \textit{exec} обязательно должна возвращать результат --- это
может быть либо результат запроса, либо статус выполнения
команды. В случае, если запрашиваемой команды не существует, клиенту
возвращается код ошибки и список доступных команд.
\subsection{Контроллер }

Контроллер --- это <<ядро>> плеера, его серверная часть. Именно он проигрывает музыку и
отвечает на запросы от клиентов. Контроллер предоставляет доступ к
сервисам модели и интерфейсы управления для клиентов. При запуске
контроллер создает точку подключения и ожидает запросы клиентов.
Контроллер реализован как традиционный процесс-демон (Daemon)(\textit{Daemon}).  При
запуске контроллер создает точку подключения и ожидает запросы клиентов.

К функциям контроллера относятся обработка запросов клиентов
(проиграть трек, пауза) и подписка клиентов на события(сменился трек,
трек приостановлен). Клиент подключается по протоколу
\textit{tcp}. Это может быть соединение на основе запроса/ответа или
постоянное подключение.

% \addimghere{controller.eps}{0.5}{Контроллер}

Для иерархических связей в сервере используется каноничная для
\textit{JavaScript} прототипная модель.  Прототипное программирование
--- стиль объектно-ориентированного программирования, при котором
отсутствует понятие класса, а наследование производится путём
клонирования существующего экземпляра объекта --- прототипа\cite{wiki:prototype}.

Для создания сервера используется базовый прототип в котором реализованы функции
для формирования сообщений, их обработки и отправки. При создании экземпляра
сервера базовый прототип расширяется массивом команд, которые он может
обрабатывать.

\subsection{Модель}

Модель --- это компонент, осуществляющий запросы к API
\textit{Soundcloud}. В логике модели скрыты функции получения и
обработки запрашиваемых данных.

В приложении используется вариант пассивной архитектуры MVC: модель не
подписывает каких-либо клиентов на изменение данных, все
взаимодействия происходят по методу запрос-ответ между моделью и
контроллером. Каждый элемент модели это команда, аналогичная по
структуре команде контроллера.  % \addimghere{model.eps}{0.5}{Модель}

\textit{Souncloud} предоставляет разработчикам приложений так
называемое REST API для доступа к контенту. 
Это традиционный метод доступа к ресурсам сети Интернет --- запрос
формируется в виде URL-адреса с параметрами поиска требуемых данных:

\begin{jscode}
  http://api.soundcloud.com/tracks.json?&tag=rock&order=hotness
\end{jscode}

В модели определена команда \textit{getTaggedList}, принимающая в
качестве параметра музыкальный жанр. С помощью этой команды контроллер
получает список композиций для воспроизведения, их адрес
воспроизведения, комментарии к композиции, её длительность и
информацию об авторе.

\subsection{Представление}

Представление является расширяемой частью используемой архитектуру. В проекте
+реализован консольный клиент к разработанному серверу.

Клиентское приложение может использовать весь функционал сервера, либо только часть
доступных функций. Для интеграции различных расширений сервера и клиента
используется следующий подход. Клиент запрашивает список команд,
доступных на сервере(getCommandList). Получив список команд клиент
оценивает набор функций, который сможет предоставить 
пользователю и принимает решение о запуске своих модулей, отображении
элементов интерфейса и о запуске приложения в целом.

Клиент может подписаться на интересующие его события, а может лишь выполнить одну
команду и завершить работу. Клиент может и не требовать никаких команд
вовсе, лишь принять список команд и выполнить команду пользователя из
командной строки. Это самый простой тип клиентов, представляющих из
себя <<адаптер>> текстового интерфейса доступа к контроллеру.

% Для облегчения создания таких клиентов удобно иметь библиотеку,
% реализующую запросы к контроллеру для нужного клиенту
% протокола. Например, библиотека REST over HTTP, WebSocket,
% IPC и пр. Также, для разработки приложений могут использоваться
% различные языки программирования. Поэтому подразумевается, что каждый
% модуль, реализающий абстракцию протокола, имеет в комплекте поставки
% также и реализацию клиентской библиотеки.  Такая библиотека служит
% <<мостом>> для приложений--клиентов, позволяя разработчику
% сконцентрироваться на программировании логики и поведения интерфейса,
% не отвлекаясь на реализацию взаимодействия.



В ходе работы был реализован клиент с текстовым интерфейсом. Программа
клиента состоит из описания компонентов интерфейса и функций обработки
ввода пользователя. При запуске или изменении размеров окна эмулятора
терминала производится расчет размеров, компоновка компонентов
интерфейса на экране и инициализация обработчиков событий
пользовательского ввода. Обработчики событий могут отправить серверу
команду, активировать или спрятать элементы интерфейса.

Также есть обработчики сообщений сервера и объект запроса. При
инициализации обработчики подписываются на интересующие клиент
оповещения от сервера: переключение композиции, смена статуса
воспроизведения, изменение текущей временной отметки проигрываемой
композиции и обновляют соответствующие элементы интерфейса по мере их
поступления. Объект запроса посылает текстовые команды серверу и
обрабатывает ответ. Для обработки и оповещений и запросов используются одни и
те же функции, так как получаемые сообщения в обоих случаях также идентичны, как и
логика их обработки.


\addimghere{client-server-activity}{0.75}{Диаграмма деятельности клиента}

Несколько слов о реализации вывода комментариев. Проигрыватель
воспроизводит композиции из сети интернет. Соединение с интернетом не
всегда стабильно, поэтому необходимо четко контролировать процесс
вывода комментария в точности в тот момент, когда он был написан, ---
это является ключевой функцией приложения.

В клиент-серверном плеере MPD нет необходимости выводить комментарии и музыка проигрывается с
локальных носителей. Поэтому в клиентах к MPD нет синхронизации
времени с сервером, подсчет ведется внутри клиента и синхронизация
происходит только если произойдет событие смены композиции, остановки
или начала воспроизведения. В комментарии к исходным файлам клиента
\textit{Ncmpcpp} эта функция называется <<режим Idle>>. За счет сетевой
передачи сигнала могут происходить небольшие смещения, но это не
критично для пользователя. Композиции хранятся на локальном диске,
поэтому вероятность задержек воспроизведения очень мала.

В случае веб-стриминга все наоборот. Экспериментальным путем
выяснено, что частота синхронизации текущей позиции в треке на клиенте
и сервере три раза в секунду достаточна для предотвращения
<<убегания>> комментариев вперед или заметных задержек. Такой подход
создает нагрузку на сеть, но она не существенна для локальных сетей,
на работу в которых расчитано приложение, в то время как алгоритм
реализации остается простым и понятным, а пользовательский
интерфейс комментариев ведет себя в соответсвии с ожиданиями
пользователя.




