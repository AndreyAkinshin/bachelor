\subsection{Модель}

Модель --- это компонент, осуществляющий запросы к API
\textit{Soundcloud}. В логике модели скрыты функции получения и
обработки запрашиваемых данных.

В приложении используется вариант пассивной архитектуры MVC: модель не
подписывает каких-либо клиентов на изменение данных, все
взаимодействия происходят по методу запрос-ответ между моделью и
контроллером. Каждый элемент модели это команда, аналогичная по
структуре команде контроллера.  % \addimghere{model.eps}{0.5}{Модель}

\textit{Souncloud} предоставляет разработчикам приложений так
называемое REST API для доступа к контенту. 
Это традиционный метод доступа к ресурсам сети Интернет --- запрос
формируется в виде URL-адреса с параметрами поиска требуемых данных:

\begin{jscode}
  http://api.soundcloud.com/tracks.json?&tag=rock&order=hotness
\end{jscode}

В модели определена команда \textit{getTaggedList}, принимающая в
качестве параметра музыкальный жанр. С помощью этой команды контроллер
получает список композиций для воспроизведения, их адрес
воспроизведения, комментарии к композиции, её длительность и
информацию об авторе.
